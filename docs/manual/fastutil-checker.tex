\htmlhr
\chapterAndLabel{Enhanced For Checker: avoid use of the enhanced \<for> statement}{enhancedfor-checker}

The Enhanced For Checker warns you if you use an enhanced \<for> statement
(sometimes called a ``foreach loop'') on an iterator that should be
accessed in a different way.

An example is the fastutil library (\url{https://fastutil.di.unimi.it/}).
It provides iterators such as
\href{https://fastutil.di.unimi.it/docs/it/unimi/dsi/fastutil/ints/IntListIterator.html}{\<IntListIterator\>}
in the \< next()> method is deprecated:  you should use \<nextInt> instead,
which avoids an unnecessary boxing and perhaps unboxing operation.  If you use the
method \<next()> explicitly in your code, the Java compiler will warn you
about use of the deprecated method.  However, if you use an enhanced for
statement (sometimes called a ``foreach loop''), there is no warning:

\begin{Verbatim}
  foreach (int i : myIntList) { ... }
\end{Verbatim}

\noindent
This construct is unnecessarily inefficient because of extra boxing and unboxing.
The Enhanced For Checker will warn you about this usage.

To run the Enhanced For Checker, supply the
\code{-processor org.checkerframework.checker.enhancedfor.EnhancedForChecker}
command-line option to javac.


\sectionAndLabel{Enhanced For annotations}{enhancedfor-annotations}

These qualifiers make up the Enhanced For type system.

\begin{description}

\item[\refqualclass{checker/enhancedfor/qual}{EnhancedForOk}]
  applies to an \<Iterator> that is acceptable to use in an enhanced \<for>
  statement.
  It also applies to an \<Iterable> whose iterator is \<@EnhancedForOk>.
  It is the default annotation for \<Iterable>s and \<Iterator>s.

\item[\refqualclass{checker/enhancedfor/qual}{EnhancedForForbidden}]
  applies to an \<Iterator> that is not acceptable to use in an enhanced \<for>
  statement.
  It also applies to an \<Iterable> whose iterator is \<@EnhancedForForbidden>.
  It applies to collections from the fastutil library.

\item[\refqualclass{checker/enhancedfor/qual}{EnhancedForUnknown}]
  indicates that it is not known at compile time whether it is acceptable
  to use the iterator in an enhanced \<for> statement.

\item[\refqualclass{checker/enhancedfor/qual}{EnhancedForBottom}]
  is the bottom qualifier.  It represents the \<null> value.  Programmers
  should rarely write it.

\end{description}

TODO:
Figure~\ref{fig-enhancedfor-hierarchy} shows the subtyping hierarchy of the
Enhanced For Checker's qualifiers.

\begin{figure}
\includeimage{enhancedfor}{9cm}
\caption{The subtyping relationship of the Enhanced For Checker's qualifiers.
  The type qualifiers are applicable to \<Iterable>, \<Iterator>, and
  their subtypes.  Qualifiers in gray are used internally by the type
  system but should never be written by a programmer.}
\label{fig-enhancedfor-hierarchy}
\end{figure}


% LocalWords:  enhancedfor foreach loop'' fastutil IntListIterator nextInt
% LocalWords:  unboxing EnhancedForOk EnhancedForForbidden
% LocalWords:  EnhancedForUnknown EnhancedForBottom
