rg\htmlhr
\chapterAndLabel{Boxing Checker: avoid use of the enhanced \<for> statement}{boxing-checker}

The Boxing Checker warns you if you use an enhanced \<for> statement
(sometimes called a ``foreach loop'') on an iterator that should be
accessed in a different way.
In the future, it will warn about additional performance problems caused by
unnecessary boxing.

An example is the JDK's
\sunjavadoc{java.base/java/util/PrimitiveIterator.html}{\code{PrimitiveIterator}}
and libraries that use or extend it such as the fastutil library (\url{https://fastutil.di.unimi.it/}).
In iterators such as 
\href{https://fastutil.di.unimi.it/docs/it/unimi/dsi/fastutil/ints/IntListIterator.html}{\<IntListIterator\>}
in the \<next()> method is deprecated:  you should use \<nextInt()> instead,
which avoids an unnecessary boxing and perhaps unboxing operation.  If you use the
method \<next()> of \<IntListIterator> explicitly in your code, the Java compiler will warn you
about use of the deprecated method.  (Not all subclasses of
\<PrimitiveIterator> deprecate \<next()>, however.)
However, if you use an enhanced \<for> statement (sometimes called a ``foreach
loop''), there is no warning:

\begin{Verbatim}
  foreach (int i : myIntList) { ... }
\end{Verbatim}

\noindent
This construct is unnecessarily inefficient because of extra boxing and unboxing.
The Boxing Checker will warn you about this usage.

To run the Boxing Checker, supply the
\code{-processor org.checkerframework.checker.boxing.BoxingChecker}
command-line option to javac.


\sectionAndLabel{Enhanced For annotations}{enhancedfor-annotations}

These qualifiers make up the Enhanced For type system.

\begin{description}

\item[\refqualclass{checker/boxing/qual}{EnhancedForOk}]
  applies to an \<Iterator> that is acceptable to use in an enhanced \<for>
  statement.
  It also applies to an \<Iterable> whose iterator is \<@EnhancedForOk>.
  It is the default annotation for \<Iterable>s and \<Iterator>s.

\item[\refqualclass{checker/boxing/qual}{EnhancedForForbidden}]
  applies to an \<Iterator> that is not acceptable to use in an enhanced \<for>
  statement.
  It also applies to an \<Iterable> whose iterator is \<@EnhancedForForbidden>.
  It applies to collections from the fastutil library.

\item[\refqualclass{checker/boxing/qual}{EnhancedForUnknown}]
  indicates that it is not known at compile time whether it is acceptable
  to use the iterator in an enhanced \<for> statement.

\item[\refqualclass{checker/boxing/qual}{EnhancedForBottom}]
  is the bottom qualifier.  It represents the \<null> value.  Programmers
  should rarely write it.

\end{description}

TODO:
Figure~\ref{fig-enhancedfor-hierarchy} shows the subtyping hierarchy of the
Boxing Checker's qualifiers.

\begin{figure}
\includeimage{enhancedfor}{9cm}
\caption{The subtyping relationship of the Boxing Checker's qualifiers.
  The type qualifiers are applicable to \<Iterable>, \<Iterator>, and
  their subtypes.  Qualifiers in gray are used internally by the type
  system but should never be written by a programmer.}
\label{fig-enhancedfor-hierarchy}
\end{figure}


\sectionAndLabel{Run-time checks for JDK classes}{boxing-jdk-runtime}

You can get run-time warnings about unnecessary unboxing operations in the
JDK by running Java like this:

\begin{Verbatim}
java -Dorg.openjdk.java.util.stream.tripwire=true MyClass ...
\end{Verbatim}

The Boxing Checker warns you at compile time about all possible
misuses, including misuses in your own code.

% Note that java.util.stream.Tripwire is a package-private class, so only
% the JDK can use it.



% LocalWords:  enhancedfor foreach loop'' fastutil IntListIterator nextInt
% LocalWords:  unboxing EnhancedForOk EnhancedForForbidden
% LocalWords:  EnhancedForUnknown EnhancedForBottom
