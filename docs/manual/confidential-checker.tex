\htmlhr
\chapterAndLabel{Confidential Checker}{confidential-checker}

The Confidential Checker helps identify insecure software implementations that may
lead to \href{https://cwe.mitre.org/data/definitions/1417.html}{sensitive information
exposure}. These vulnerabilities arise when developers allow information such as PII,
passwords, or local file paths to be passed to potentially public-facing sinks, which
include website and app user interfaces, log files, error messages, and URL targets.

Exposure may occur mistakenly or with a benign objective in mind. For instance, a developer
may unwittingly write code that exposes sensitive information, or they may output
sensitive information to error messages with the intention of supplying helpful
troubleshooting direction; the Confidential Checker covers both of these cases by
guaranteeing that no values marked as sensitive information will be passed to a
location marked as public-facing.

However, the Confidential Checker cannot guarantee the correctness or completeness
of the annotations themselves. In other words, the onus of comprehensively annotating
sensitive information falls solely upon the developer. The Confidential Checker does
provide preliminary annotations for certain public-facing sinks, but developers should
always verify that any sinks used in their code are annotated appropriately.

To run the Confidential Checker, supply the
\code{-processor ConfidentialChecker}
or
\code{-processor org.checkerframework.checker.confidential.ConfidentialChecker}
command-line option to javac.


\sectionAndLabel{Confidential annotations}{confidential-annotations}

The Confidential Checker type system uses the following annotations:

\begin{description}
\item[\refqualclass{checker/confidential/qual}{Confidential}]
    indicates a value that will not be exposed to end users or a sink that
    will not be able to be accessed by end users.
\item[\refqualclass{checker/confidential/qual}{NonConfidential}]
    indicates a value that may be exposed to end users, or a location that
    may be accessed by end users. It is the default qualifier.
\item[\refqualclass{checker/confidential/qual}{UnknownConfidential}]
    indicates a value whose confidentiality is unknown.
    \code{@UnknownConfidential} is a supertype of \code{@Confidential}
    and \code{@NonConfidential}.
\item[\refqualclass{checker/confidential/qual}{PolyConfidential}]
    indicates qualifier polymorphism. For a description of qualifier polymorphism,
    see Section~\ref{method-qualifier-polymorphism}.
\item[\refqualclass{checker/confidential/qual}{BottomConfidential}]
    is the bottom qualifier. Programmers rarely need to write it.
\end{description}

The subtyping hierarchy of the Confidential Checker's qualifiers is shown in
Figure~\ref{fig-confidential-hierarchy}.

\begin{figure}
\includeimage{confidential}{5.5cm}
\caption{The subtyping relationship of the Confidential Checker's qualifiers.
  Qualifiers in gray are used internally by the type system but should never
  be written by a programmer.}
\label{fig-confidential-hierarchy}
\end{figure}

Although parallels could be drawn between the Confidential Checker's
\code{@Confidential} annotation and the Taint Checker's \code{@Tainted}
annotation (see Section~\ref{tainting-annotations}), it should be noted that
there is no subtyping relationship between \code{@Confidential} and
\code{@NonConfidential}. However, the external behavior of these annotations
can strongly resemble that of a supertype and subtype, as casts from the latter
to the former are always permitted (see Section~\ref{confidential-checks}). As
a result, while the Confidential Checker (correctly) does not allow
\code{@Confidential} values to be passed to \code{@NonConfidential} sinks, it does
allow \code{@NonConfidential} values in \code{@Confidential} sinks.

\sectionAndLabel{What the Confidential Checker checks}{confidential-checks}

Concatenation of \code{@Confidential} and \code{@NonConfidential} values is
always dominated by \code{@Confidential}. In other words, concatenation of
\code{@Confidential} and \code{@NonConfidential} values are most narrowly typed
as follows:

\begin{Verbatim}
    @Confidential + @Confidential = @Confidential
    @Confidential + @NonConfidential = @Confidential
    @NonConfidential + @Confidential = @Confidential
    @NonConfidential + @NonConfidential = @NonConfidential
\end{Verbatim}

However, casts of any type in the Confidential hierarchy to \code{@Confidential}
are always allowed as they do not pose a risk of sensitive information exposure:
\code{@NonConfidential} to \code{@Confidential} casts only increase stringency,
while \code{@UnknownConfidential} is subtyped by \code{@Confidential},
\code{@NonConfidential} and can thus be safely cast. This allowance was motivated
by a high percentage of \code{@SuppressWarnings} annotations being written to allow
casts from \code{@NonConfidential} to \code{@Confidential}.

\sectionAndLabel{Library annotations}{confidential-library-annotations}

As mentioned prior, the Confidential Checker provides preliminary annotations
for certain public-facing sinks either in the form of stub files or within the
Checker Framework's annotated JDK. For instance, the \code{TextView} class for Android
apps requires that a \code{@NonConfidential CharSequence} should be passed to
its \code{setText} method to avoid displaying sensitive information on the user's
viewport.

If necessary, developers can write methods to vet \code{@Confidential} or
\code{@UnknownConfidential} values and determine which, if any, parts of the
value can be safely cast to \code{@NonConfidential} under a \code{@SuppressWarnings}
annotation and passed to user-facing sinks.

It should be noted that, since \code{@NonConfidential} is the default annotation,
developers must be particularly attentive to explicitly annotating their variables
to avoid accidentally passing a secret value to a public-facing sink. For this reason,
it may seem insecure to set \code{@NonConfidential} as the default rather than
\code{@UnknownConfidential} or \code{@Confidential}. However, this implementation
significantly decreases the number of \code{@SuppressWarnings} annotations (and
overall annotations)that the developer must write, as in practice, values are more
often \code{@NonConfidential} than \code{@Confidential}.
