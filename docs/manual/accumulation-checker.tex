\htmlhr
\chapterAndLabel{Building an accumulation checker}{accumulation-checker}

%% This chapter should appear after the "creating a checker" chapter, or perhaps as part of it,
%% once accumulation support is complete.

This chapter describes how to build a checker for an accumulation analysis.
If you want to \emph{use} an existing checker, you do not need to read this chapter.

An \emph{accumulation analysis} is a program analysis where the
analysis abstraction is a monotonically increasing set --- that is, the
analysis learns new facts, and facts are never retracted.
Typically, some operation in code is legal
only when the set is large enough --- that is, the estimate has accumulated
sufficiently many facts.

The Called Methods Checker (\chapterpageref{called-methods-checker})
is an accumulation analysis.
The
\href{https://github.com/typetools/checker-framework/blob/master/framework/src/test/java/org/checkerframework/framework/testchecker/testaccumulation/TestAccumulationChecker.java}{Test Accumulation Checker}
is a simplified version of the Called Methods Checker that you can also use as a model.

Accumulation analysis is a special case of typestate analysis in which
(1) the order in which operations are performed does not affect what is subsequently legal,
and (2) the accumulation does not add restrictions; that is, as
more operations are performed, more operations become legal.
Unlike a traditional typestate analysis, an accumulation analysis does
not require an alias analysis for soundness. It can therefore be implemented
as a flow-sensitive type system.

The rest of this chapter assumes you have read how to create a checker
(\chapterpageref{creating-a-checker}).


\paragraphAndLabel{Defining type qualifiers}{accumulation-qualifiers}
Define 2 or 3 type qualifiers.

\begin{itemize}
\item
The ``accumulator'' type qualifier has a single argument: a \<String[]> named
\<value> that defaults to the an empty array.  Note that the Checker Framework's
support for accumulation analysis requires you to accumulate a string representation
of whatever you are accumulating. For example, when accumulating which methods have
been called, you might choose to accumulate method names.

The accumulator
qualifier should have no supertypes (\<@SubtypeOf({})>) and should
be the default qualifier in the hierarchy (\<@DefaultQualifierInHierarchy>).

An example of such a qualifier can be found in the Checker Framework's tests:
\href{https://github.com/typetools/checker-framework/blob/master/framework/src/test/java/org/checkerframework/framework/testchecker/testaccumulation/qual/TestAccumulation.java}{TestAccumulation.java}.

\item
Define a bottom type, analogous to
\href{https://github.com/typetools/checker-framework/blob/master/framework/src/test/java/org/checkerframework/framework/testchecker/testaccumulation/qual/TestAccumulationBottom.java}{TestAccumulationBottom.java}.
It should take no arguments, and should be a subtype of the accumulator type you defined earlier.

\item
Optionally, define a predicate annotation, analogous to
\href{https://github.com/typetools/checker-framework/blob/master/framework/src/test/java/org/checkerframework/framework/testchecker/testaccumulation/qual/TestAccumulationPredicate.java}{TestAccumulationPredicate.java}.
It must have a single argument named \<value> of type \<String>.
The predicate syntax supports
\begin{itemize}
\item \<||> disjunctions
\item \<\&\&> conjunctions
\item \<!> logical complement.  \<"!x"> means
``it is not true that \<x> was definitely accumulated'' or, equivalently, ``there is no path on which \<x> was accumulated''.
Note that this does \textbf{not} mean ``\<x> was not accumulated'' --- it is not a violation of the specification \<"!x"> if \<x> is accumulated
on some paths, but not others.
\item \<(...)> parentheses for precedence
\end{itemize}

\end{itemize}

\paragraphAndLabel{Setting up the checker}{accumulation-setup}

Define a new class that extends \refclass{common/accumulation}{AccumulationChecker}.
It does not need any content.

Define a new class that extends \refclass{common/accumulation}{AccumulationAnnotatedTypeFactory}.
You must create a new constructor whose only argument is a \refclass{common/basetype}{BaseTypeChecker}.
Your constructor should call one of the \<super> constructors defined in
\refclass{common/accumulation}{AccumulationAnnotatedTypeFactory} (which one depends on whether or not
you defined a predicate annotation).


\paragraphAndLabel{Adding accumulation logic}{accumulation-accumulating}

Define a class that extends \refclass{common/accumulation}{AccumulationTransfer}.
To update the estimate of what has been accumulated, override some method in
\refclass{framework/flow}{CFAbstractTransfer} to call
\refmethodterse{common/accumulation}{AccumulationTransfer}{accumulate}{-org.checkerframework.dataflow.cfg.node.Node-org.checkerframework.dataflow.analysis.TransferResult-java.lang.String...-}.

For example, to accumulate the names of methods called, a checker would override
\refmethod{framework/flow}{CFAbstractTransfer}{visitMethodInvocation}{-org.checkerframework.dataflow.cfg.node.MethodInvocationNode-org.checkerframework.dataflow.analysis.TransferInput-} to:
call \<super> to get a \<TransferResult>, compute the method name from the \<MethodInvocationNode>,
and then call \<accumulate>.


\paragraphAndLabel{Enforcing program properties}{accumulation-enforcing}

At this point, your checker ensures that all annotations are consistent
with one another and the source code, and it flow-sensitively refines the
annotations.
Te enforce properties in the code being type-checked, write type rules
(Section~\ref{creating-extending-visitor}) that are specific to your type
system.


\sectionAndLabel{Publications}{accumulation-publications}

The paper
\href{https://homes.cs.washington.edu/~mernst/pubs/accumulation-analysis-ecoop2022.pdf}{``Accumulation
  Analysis''}~\cite{KelloggSSE2022} describes theoretical properties of
accumulation analysis.  The papers
\href{https://homes.cs.washington.edu/~mernst/pubs/object-construction-icse2020.pdf}{``Verifying
  Object Construction''}~\cite{KelloggRSSE2020} and
\href{https://homes.cs.washington.edu/~mernst/pubs/resource-leak-esecfse2021.pdf}{``Lightweight
  and modular resource leak verification''}~\cite{KelloggSSE2021} describe
specific accumulation analyses.


% LocalWords:  SubtypeOf TestAccumulation TestAccumulationBottom
% LocalWords:  TestAccumulationPredicate AccumulationChecker
% LocalWords:  AccumulationAnnotatedTypeFactory BaseTypeChecker
% LocalWords:  AccumulationTransfer CFAbstractTransfer TransferResult
% LocalWords:  visitMethodInvocation MethodInvocationNode
